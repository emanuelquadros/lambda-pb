\documentclass[11pt]{article}
\usepackage[T1]{fontenc}
\usepackage{graphicx}
\usepackage{longtable}
\usepackage{float}
\usepackage{wrapfig}
\usepackage{rotating}
\usepackage[normalem]{ulem}
\usepackage[fleqn]{amsmath}
\usepackage{textcomp}
\usepackage{marvosym}
\usepackage{wasysym}
\usepackage{amssymb}
\usepackage{hyperref}
\tolerance=1000
\usepackage[backend=biber,bibstyle=biblatex-sp-unified,citestyle=authoryear,maxcitenames=3,doi=false,url=false]{biblatex}
\addbibresource{~/Dropbox/research/ling.bib}
\usepackage[brazil]{babel}
\usepackage[margin=1in]{geometry}
\usepackage{bookmark}
\usepackage{stmaryrd}
\usepackage{amssymb}
\usepackage{amsmath}
\usepackage{tabularx}
\usepackage{multirow}
\usepackage{soul, color}
\soulregister\fullcite7
\usepackage{fontspec}
\setmainfont{Linux Libertine O}
\usepackage[libertine]{newtxmath}
\usepackage{mdframed}
\usepackage{csquotes}
\renewcommand{\baselinestretch}{1.5}
\usepackage{tabto}
\usepackage{gb4e}
\noautomath
\newcommand{\den}[1]{\ensuremath{\llbracket #1\rrbracket}}
\hypersetup{hidelinks=true}
\author{}
\date{Versão de \today}
\title{Revisão de teoria dos conjuntos para semanticistas}

\begin{document}
\maketitle

\section{O que é um conjunto?}
\label{sec:conjuntos}

Um conjunto é uma coleção de objetos quaisquer. Esses objetos são os \textbf{elementos} (ou \textbf{membros}) do respectivo conjunto.

Seguem alguns exemplos de conjuntos. Esses exemplos serão utilizados no restante deste texto.

\begin{align*}
  A & = \{2, 3, 4\}\\
  B &= \{\mathrm{Frege}, \mathrm{Montague}, \mathrm{Partee}, \mathrm{Russell}\}\\
  C &= \{2, 3, \{4\}, \mathrm{Partee}\}\\
  D &= \{\text{todas as sentenças do português}\}\\
  E &= \{4, 3, 2\}
\end{align*}

\section{Relações importantes definidas sobre conjuntos}
\label{sec:relacoes}

\subsection{Pertinência (relação entre elementos e conjuntos)}
\label{sec:pertinencia}

\begin{align*}
  4 \in A & \hspace*{\fill} & 4 \text{ é um elemento de (pertence a) } A.\\
  4 \not\in C & \hspace*{\fill} & 4 \text { não é um elemento de } C.\\
  \{4\} \not\in A & & \{4\}\text { não é um elemento de } A.\\
  \{4\} \in C & & \{4\}\text { é um elemento de } C.
\end{align*}

\textbf{Importante:} \(4\) e \(\{4\}\) são objetos diferentes!

\subsection{Continência (relação entre conjuntos)}
\label{sec:continencia}

\(X\) é um \textbf{subconjunto} de \(Y\) se e somente se todos os elementos de \(X\) também são elementos de \(Y\).

\begin{align*}
  A \subseteq \mathbb{N} & & A \text{ é um subconjunto do (está contido no) conjunto dos números naturais.}\\
  A \subseteq E & & A \text{ é um subconjunto de } E.\\
  E \subseteq A & & E \text{ é um subconjunto de } A.\\
  A \subset \mathbb{N} & & A \text{ é um subconjunto próprio de (está estritamente contido em) } \mathbb{N}.\\
  A \not\subset E & & A \text{ não é um subconjunto próprio de } E.
\end{align*}

\subsection{Igualdade entre conjuntos}
\label{sec:igualdade}

Conjuntos são definidos somente por seus elementos. Dois conjuntos são idênticos se e somente se eles possuem os mesmos elementos. Essa definição de identidade tem como consequência que \(A\) e \(E\) acima são o mesmo conjunto, mesmo que seus elementos tenham sido apresentados em ordens distintas na seção \ref{sec:conjuntos}.

Um fato importante, que se segue dessa definição de identidade: para quaisquer conjuntos \(X\) e \(Y\), se \(X\subseteq Y\) e \(Y\subseteq X\), então \(X = Y\).

\subsection{Conjunto potência (ou conjunto de partes)}
\label{sec:conjunto-potencia}

Um conjunto muito especial que pode ser formado a partir de qualquer conjunto é o conjunto potência. Dado um conjunto \(X\), o conjunto potência de \(X\), ou \(\mathcal{P}(X)\), é o conjunto de todos os subconjuntos de \(X\).

Assim, se \(A = \{2, 3, 4\}\), então \(\mathcal{P}(A) = \{\emptyset, \{2\}, \{3\}, \{4\}, \{2, 3\}, \{3, 4\}, \{2, 4\}, \{2, 3, 4\}\}\).

Duas conclusões seguem da definição de subconjunto apresentada na seção \ref{sec:continencia}:
\begin{itemize}
\item Todo conjunto é um subconjunto de si mesmo.
\item O conjunto vazio, denotado por \(\emptyset\), é um subconjunto de qualquer conjunto.

  Se você não se convence disso, tente mostrar como isso poderia ser falso; isto é, dado um conjunto \(X\) qualquer, tente mostrar que \(\emptyset\) não é um subconjunto de \(X\), partindo da definição de subconjunto. Esse tipo de raciocínio é muito usado em provas por contradição (ou \emph{reductio ad absurdum}).
\end{itemize}

\end{document}

%%% Local Variables:
%%% mode: latex
%%% TeX-master: t
%%% End:
